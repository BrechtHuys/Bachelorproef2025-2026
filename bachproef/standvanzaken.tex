\chapter{\IfLanguageName{dutch}{Stand van zaken}{State of the art}}%
\label{ch:stand-van-zaken}

% Tip: Begin elk hoofdstuk met een paragraaf inleiding die beschrijft hoe
% dit hoofdstuk past binnen het geheel van de bachelorproef. Geef in het
% bijzonder aan wat de link is met het vorige en volgende hoofdstuk.

% Pas na deze inleidende paragraaf komt de eerste sectiehoofding.
\section{Introductie}
Mainframes nemen nog altijd een merkwaardige positie in binnen hedendaagse IT-landschappen. Ze worden beschreven als relieken, logge machines uit een ander tijdperk, maar in grote banken, overheidsinstanties en industriële conglomeraten blijven ze onmisbaar\autocite{Gangula2026}. Op een gemiddelde werkdag kan een enkel systeem stilletjes miljoenen betalingsinstructies afhandelen, pensioenrechten bijwerken of voorraadstromen zonder onderbreking reconciliëren \autocite{BharathiVenkataramanapa2026}. Beschikbaarheid is geen luxe in dergelijke omgevingen; het is een verwachting. Integriteit wordt verondersteld. Vertrouwelijkheid, indien geschonden, zou niet alleen de organisatie in verlegenheid brengen, maar zou ook kunnen leiden tot regelgevende maatregelen en publiek wantrouwen.
Gezien dat gewicht, wordt het beheer van toegangsrechten minder een administratieve routine en meer een belangrijke noodzaak. Wanneer privileges te breed worden toegewezen of simpelweg achterblijven na een rolverandering, zijn de risico's niet abstract. Een overgekwalificeerde operations clerk zou onbedoeld zicht kunnen krijgen op klantdatasets; een voormalige ontwikkelaar zou updatebevoegdheid kunnen houden op productiebibliotheken. Deze situaties veroorzaken niet onmiddellijk een catastrofe, maar ze accumuleren spanning in het systeem. Ze veringewikklen audits, verduisteren verantwoordelijkheid, en maken, in een zekere omstandigheden, wangedrag gemakkelijker dan het zou moeten zijn \autocite{BharathiVenkataramanapa2026}.

Binnen IBM z/OS wordt toegangscontrole voornamelijk afgedwongen via de Resource Access Control Facility (RACF)\autocite{Troy2026}. RACF structureert autorisatie rond gebruikersprofielen, groepshiërarchieën en resource-definities datasets, transacties, commando's, elk gelaagd met permissies die uitdrukkelijk  geërfd kunnen zijn \autocite{Pomerantz2026}. Het model is tegelijk elegant en labyrintisch. De granulariteit ervan stelt een beveiligingsbeheerder in staat om opmerkelijk precieze beleidsregels uit te drukken. Tegelijkertijd kan de wisselwerking tussen groepslidmaatschap en directe toewijzingen de effectieve rechten van een  gebruiker vertroebelen. Bepalen wie er daadwerkelijk toegang heeft tot een gevoelige dataset vereist  het volgen van overlappende erfpaden \autocite{BharathiVenkataramanapa2026}.

Het principe van de Least privilege, dat  wordt ingeroepen in het discours over Identiteits- en Toegangsbeheer, lijkt in theorie eenvoudig. Geef alleen wat nodig is; intrek wat niet nodig is \autocite{Hecht2026}. Toch compliceren mainframe-omgevingen dit ideaal. Gebruikers blijven tientallen jaren in dienst. Verantwoordelijkheden verschuiven geleidelijk, soms informeel. Noodtoegang wordt verleend in momenten van operationele drukte en niet altijd met dezelfde urgentie ingetrokken. In de loop van de tijd stapelen rechten zich op. Wat begon als een tijdelijke maatregel, vestigt zich in blijvende vorm. Het systeem vergeet niet tenzij iemand het dwingt \autocite{Hecht2026} \autocite{BharathiVenkataramanapa2026}.

RACF is in dit opzicht noch het probleem noch de oplossing. Het biedt de mechanismen voor gedisciplineerde controle, maar het handhaaft geen terughoudendheid \autocite{Troy2026}. Of het principe van least privilege wordt gerealiseerd, hangt af van governanceprocessen, documentatiepraktijken en de bereidheid om lang bestaande configuraties in twijfel te trekken \autocite{Gangula2026}. Men zou kunnen betogen dat de technische mogelijkheden de procedurele controle hebben ingehaald; een ander zou kunnen stellen dat de complexiteit van de organisatorische realiteit onvermijdelijk weerstand biedt tegen strikte minimalisering van toegang. Beide standpunten bevatten enige waarheid.

Om die reden kan de detectie van privilege creep niet alleen op intuïtie vertrouwen. Het vereist een methode die systematisch, reproduceerbaar en in staat is om de gelaagde structuren in RACF-profielen te ontrafelen \autocite{Hecht2026} \autocite{BharathiVenkataramanapa2026}. Door gebruikersdefinities, groepslidmaatschappen en resource-autorisaties gezamenlijk te analyseren, wordt het mogelijk om permissies te identificeren die functioneel overbodig of historisch contingent zijn \autocite{Hecht2026} \autocite{BharathiVenkataramanapa2026}. Zulke analyses versterken niet alleen de beveiligingshouding. Het verduidelijkt verantwoordelijkheden, vereenvoudigt audits en herstelt een  mate van transparantie in omgevingen die, door jaren van geleidelijke verandering, ondoorzichtig zijn geworden \autocite{BharathiVenkataramanapa2026} \autocite{Gangula2026}.

\section{IBM z/OS en RACF Security Overview}
\subsection{Wat zijn IBM z/OS mainframes?}

Een IBM z/OS-systeem in werking tegenkomen is, in zekere zin, getuige zijn van continuïteit die technisch is gemaakt. Deze machines worden  beschreven in termen van betrouwbaarheid, beschikbaarheid, onderhoudbaarheid en beveiliging, kwaliteiten die bijna ceremonieel zijn geworden in de mainframe-discussie \autocite{Dau2026}. Toch ligt achter die vertrouwde termen een praktische realiteit: ze verwerken buitengewone hoeveelheden transacties, zonder pauze, en dat doen ze onder omstandigheden waarin onderbreking niet alleen ongemakkelijk maar ook institutioneel ontwrichtend is \autocite{Borjigin2026}. Een nationale betalingsschakel kan bijvoorbeeld niet gewoon "later opnieuw proberen." Pensioenuitkeringen en interbancaire afrekeningen verwachten dat het systeem aanwezig, wakker en nauwkeurig is \autocite{Paul2026}.
De meeste z/OS-installaties draaien continu, vierentwintig uur per dag, zeven dagen per week, en verwerken miljoenen dagelijkse transacties met slechts nauwkeurig geplande onderhoudsvensters \autocite{Bernstein2026}. Hun architectuur ondersteunt batchverwerking met hoge doorvoer tijdens nachtelijke cycli, online transactieprocessing tijdens kantooruren, en het beheer van uitgebreide datasets waarvan de oorsprong tientallen jaren terug kan gaan \autocite{Dau2026}. Men zou kunnen stellen dat gedistribueerde platforms nu een vergelijkbare schaal bieden. Misschien doen ze dat. Toch lijken veel organisaties in de praktijk terughoudend om hun meest gevoelige, omzetgenererende werkbelastingen toe te vertrouwen aan omgevingen waar consistentie voornamelijk wordt bereikt door coördinatie tussen losjes gekoppelde knooppunten in plaats van door gecentraliseerde controle.

Fouttolerantie in z/OS is niet toevallig; het is ingebed in de ontwerpveronderstellingen. Redundante processors, gemirroreerde opslag, werkbelastingbeheerbeleid en herstelvoorzieningen werken samen zodat de uitval van een component niet noodzakelijkerwijs de uitval van het systeem veroorzaakt \autocite{Borjigin2026}. De architectuur anticipeert op storingen en absorbeert ze. Deze veerkracht moet echter niet geromantiseerd worden. Het hangt af van zorgvuldige configuratie, gedisciplineerd verandermanagement en beheerders die de implicaties van elke parameterwijziging begrijpen \autocite{Dau2026}. Hoge beschikbaarheid is geconfigureerd, niet geërfd.

Beveiliging is ook nauw verweven met de operationele omgeving. Via mechanismen zoals Resource Access Control Facility (RACF) integreert z/OS authenticatie en autorisatie in de kern van de systeemoperatie in plaats van ze als perifere diensten te behandelen \autocite{MarkNelson2026}. Toegangsbeslissingen regelen datasets, transacties en commando's op een niveau van granulariteit dat zelfs ervaren professionals kan verrassen \autocite{Kanvar2026}. In theorie maakt dit strikte handhaving van enterprise identity and access management-beleid mogelijk. In de praktijk kan het ook ingewikkelde configuraties creëren die moeilijk in hun geheel te overzien zijn, vooral in omgevingen die zijn gevormd door jaren van incrementele aanpassingen \autocite{Dau2026}.

De sectoren waarin deze systemen prevalent zijn - bankieren, verzekeringen, overheidsadministratie, productie - zijn niet toevallig diegene die onderworpen zijn aan rigoureuze regulatoire controle \autocite{Wagenseil2026}. Compliance-eisen, auditsporen en gegevensbeschermingsmandaten oefenen constante druk uit\autocite{Wagenseil2026}. Een vertraagde transactie of een ongeoorloofd toegangsevenement kan gevolgen hebben die verder reiken dan technische remediering naar juridische en reputatiegebieden. Voor dergelijke organisaties fungeert de mainframe vaak als een stabiliserend centrum: computationeel krachtig, voorspelbaar schaalbaar en, wanneer goed beheerd, veilig.

Toch kan het beschrijven van z/OS als de onbetwiste ruggengraat van enterprise IT de zaken te vereenvoudigen. De sterke punten zijn duidelijk, maar ze gaan gepaard met complexiteit en een  mate van institutionele afhankelijkheid die nader onderzoek rechtvaardigt. Dezelfde kenmerken die het platform veerkrachtig en veilig maken, kunnen het ondoorzichtig maken. Die ondoorzichtigheid is precies de reden waarom zorgvuldige beheersing van autorisatiecontroles, met name die via RACF worden geïmplementeerd, essentieel blijft. In een omgeving die is gebouwd voor permanentie, hebben toegangsrechten, eenmaal verleend, de neiging om aan te houden.

\subsection{Wat is RACF?}

RACF maakt het zo dat toegangrechten technisch af te dwingen via een hiërarchische en fijnmazige autorisatiestructuur. Rechten kunnen direct aan individuele gebruikers worden toegewezen of via groepsstructuren worden geërfd\autocite{Kanvar2026}. In grote organisaties leidt dit  tot historisch geëvolueerde configuraties die moeilijk te overzien zijn.
RACF fungeert als het primaire beveiligingsmechanisme binnen IBM z/OS en vormt een centraal onderdeel van Identity and Access Management (IAM) op mainframes \autocite{MarkNelson2026}. Het systeem biedt een gestructureerde manier om gebruikersaccounts, groepen en middelen te beheren, en om autorisaties af te dwingen volgens vastgestelde beleidsregels. Gebruikersprofielen bevatten basisinformatie over de identiteit, toegangsrechten en accountstatussen van individuele werknemers, terwijl groepen worden gebruikt om rechten collectief toe te wijzen, wat het beheer van grote aantallen gebruikers vereenvoudigt \autocite{Kanvar2026}.

De autorisatiestructuur van RACF is ontworpen om principes toe te passen zoals het principe van de Least privilege, waarbij gebruikers alleen de rechten krijgen die functioneel noodzakelijk zijn voor hun rol\autocite{YuandPeiyangXueandYunhanQuandLeiJu2026}. RACF ondersteunt ook het auditen en loggen van toegangsactiviteiten, waardoor organisaties kunnen voldoen aan compliance-eisen en interne controle van toegangsbeheer kunnen uitvoeren \autocite{Wagenseil2026}.

In complexe mainframe-omgevingen ontstaan echter  overlap en redundantie in permissies door historische configuraties, veranderende rolverdelingen en tijdelijke toegangsrechten. Dit kan leiden tot situaties waarin gebruikers meer rechten bezitten dan functioneel noodzakelijk, een fenomeen dat bekend staat als privilege creep\autocite{Boorman2026}. Vanwege de combinatie van hiërarchische groepsstructuren, geërfde rechten en directe toewijzingen, is het moeilijk voor beheerders om handmatig te bepalen welke toegangsrechten nog daadwerkelijk belangrijjk zijn.

Bovendien biedt RACF mechanismen voor het beheer van resourceklassen, waardoor  systeembronnen (zoals datasets, programma's en TSO-commando's) individueel kunnen worden beschermd \autocite{MarkNelson2026}. Elke resourceklasse heeft specifieke attributen en autorisatiemogelijkheden, die bijdragen aan flexibiliteit maar ook de complexiteit van toegangsbeheer verhogen \autocite{Kleuskens2026}. Voor grote organisaties is dit een uitdaging omdat veranderingen in een groep of bron indirect tientallen of honderden gebruikers kunnen beïnvloeden, wat snel kan leiden tot een verlies van overzicht.

Kortom, RACF biedt een krachtig en gedetailleerd kader voor toegangsbeheer op z/OS mainframes, maar de combinatie van flexibiliteit, hiërarchie en historisch gegroeide configuraties maakt het noodzakelijk om systematische methoden te ontwikkelen om overbodige of risicovolle machtigingen te identificeren. Dit vormt de basis voor het onderzoek naar het detecteren van privilege creep in deze thesis.

\section{Toegangsbeheeruitdagingen}
\subsection{Rolwijzigingen en gebruikerslevenscyclus}

In mainframe-omgevingen veranderen gebruikersrollen regelmatig, bijvoorbeeld als werknemers van afdeling of functie wisselen, tijdelijke projecten ondernemen of extra verantwoordelijkheden krijgen toegewezen. Bij elke rolverandering worden nieuwe toegangsrechten verleend zodat de gebruiker zijn taken kan uitvoeren. Zonder een systematische en tijdige intrekking van oudere rechten, blijven deze maar toch actief. Dit leidt tot een accumulatie van onnodige toegangsrechten, wat het risico op ongeautoriseerde toegang en privilege creep vergroot \autocite{Boorman2026}.
De levenscyclus van een gebruiker in een z/OS-omgeving omvat fasen: het aanmaken van een account, het toekennen van initiële rechten, tijdelijke of aanvullende rechten tijdens functiewijzigingen, en uiteindelijk het intrekken van rechten bij beëindiging van het dienstverband of vertrek. RACF biedt mechanismen om accounts en rechten te beheren, maar in de realiteit worden intrekkingsprocessen handmatig uitgevoerd of periodiek herzien\autocite{Kanvar2026}. Deze handmatige processen zijn foutgevoelig en leiden tot situaties waarin oude of niet langer belangrijke rechten blijven bestaan.

Het oplopende effect van herhaalde rolveranderingen in de loop van de tijd kan aanzienlijk zijn, in grote organisaties met langdurig actieve gebruikers. In dergelijke gevallen wordt het moeilijk om te bepalen welke rechten nog functioneel noodzakelijk zijn en welke overbodig zijn geworden \autocite{Dau2026}. Dit vormt een belangrijke bron van problemen met toegangsbeheer, verhoogt de complexiteit van audits en dient als een directe oorzaak voor de opkomst van privilege creep.

Bovendien kan de afwezigheid van geautomatiseerde evaluatie van gebruikersrechten leiden tot vertragingen in nalevingsrapportages en inefficiënties in periodieke toegangsbeoordelingen. Dit benadrukt het belang van een systematische en reproduceerbare methode om rechten te analyseren en te beoordelen, zodat de principes van het least privilege consistent kunnen worden toegepast\autocite{YuandPeiyangXueandYunhanQuandLeiJu2026}.

Kortom, de dynamiek van rolveranderingen en de gebruikerslevenscyclus zijn een kernfactor in de ontwikkeling van privilege creep binnen mainframe-omgevingen. Het begrijpen van deze dynamiek is essentieel voor het ontwerpen van technische methoden die overmatige privileges kunnen detecteren en de veiligheid en naleving van het systeem kunnen waarborgen\autocite{Kleuskens2026}.

\subsection{Groep hiërarchieën en geërfde machtigingen}

In RACF-omgevingen worden toegangsrechten georganiseerd via hiërarchische groepsstructuren, waarbij gebruikers lid zijn van groepen die bepaalde rechten verlenen\autocite{Kanvar2026}. Deze groepsstructuren stellen beheerders in staat om efficiënt rechten toe te wijzen aan een groep gebruikers in plaats van aan individuele accounts. Hoewel dit beheersconcept voordelen biedt op het gebied van schaalbaarheid en overzicht, brengt het ook uitdagingen met zich mee.
Een belangrijke consequentie van deze hiërarchieën is dat gebruikers rechten erven die functioneel noodzakelijk zijn voor hun huidige rol. Rechten kunnen cumulatief worden toegewezen via  lagen van groepslidmaatschap, waardoor het voor beheerders moeilijk wordt om de effectieve set rechten van een gebruiker nauwkeurig te bepalen\autocite{Kanvar2026}. Dit is problematisch in grote organisaties waar groepsstructuren historisch zijn geëvolueerd en  groepen overlappen in hun rechten\autocite{Dau2026}.

Het handmatig controleren van deze geërfde rechten is tijdrovend en foutgevoelig. Beheerders moeten niet alleen de directe rechten van een gebruiker analyseren, maar ook alle impliciete rechten die via groepslidmaatschappen zijn geërfd\autocite{Boorman2026}. Het risico bestaat dat overbodige of ongepaste rechten over het hoofd worden gezien, wat kan leiden tot verhoogde kwetsbaarheid en de accumulatie van privilege creep.

Bovendien bemoeilijkt de complexiteit van groepshiërarchieën de uitvoering van audits en toegangsbeoordelingen\autocite{Wagenseil2026}. Auditors en beveiligingsbeheerders hebben volledige inzicht nodig in zowel directe als geërfde rechten om te beoordelen of de toegewezen toegangsrechten voldoen aan het principe van de Least privilege\autocite{YuandPeiyangXueandYunhanQuandLeiJu2026}. Zonder systematische tools kan dit proces inefficiënt en inconsistent zijn, wat de betrouwbaarheid van de controles ondermijnt\autocite{Kleuskens2026}.

Kortom, de hiërarchische structuur van groepen in RACF biedt schaalvoordelen, maar creëert tegelijkertijd een complex en moeilijk te overzien netwerk van rechten. Dit vormt een belangrijke bron van overbodige of risicovolle toegangsrechten en benadrukt de noodzaak van technische methoden om effectieve rechten te analyseren en privilege creep te detecteren.

\subsection{Manual audits and operational limitations}
Traditionele audits in RACF-omgevingen zijn  arbeidsintensief en tijdrovend, omdat ze een gedetailleerde analyse vereisen van zowel individuele gebruikersrechten als groeps- en resourceconfiguraties\autocite{Wagenseil2026}. Beheerders en auditors moeten handmatig controleren welke rechten gebruikers hebben, hoe deze rechten via groepen worden overgedragen, en of deze rechten nog functioneel noodzakelijk zijn. Dit proces vereist een diepgaande kennis van de RACF-architectuur en de specifieke bedrijfsprocessen, waardoor het sterk afhankelijk is van de expertise van individuele medewerkers.
Vanwege de complexiteit van RACF-omgevingen en de  historisch gegroeide configuraties, is het moeilijk om consistente en reproduceerbare audits uit te voeren\autocite{Dau2026}. Periodieke controles kunnen daarom variëren in kwaliteit en volledigheid, afhankelijk van wie de audit uitvoert en welke gegevens op dat moment beschikbaar zijn. Dit vergroot het risico dat overmatige of ongepaste rechten over het hoofd worden gezien, waardoor gebruikers meer  toegangsrechten behouden dan functioneel noodzakelijk\autocite{Boorman2026}.

Bovendien zijn handmatige audits niet schaalbaar in grote organisaties met honderden tot duizenden gebruikers en complexe groepshiërarchieën\autocite{Kanvar2026}. Het handmatig analyseren van alle rechten en hun relaties met datasets en andere middelen vereist  tijd en middelen, wat leidt tot operationele inefficiënties. Deze beperkingen kunnen ertoe leiden dat audits minder worden uitgevoerd dan aanbevolen, waardoor privilege creep ongemerkt kan accumuleren.

Ten slotte bemoeilijkt de afhankelijkheid van individuele kennis en de handmatige aard van het proces de naleving en rapportage\autocite{Wagenseil2026}. Voor interne en externe audits is het essentieel om reproduceerbaar en objectief bewijs van toegangscontrole te leveren. Zonder geautomatiseerde tools en systematische methoden is het moeilijk om aan deze vereisten op een betrouwbare manier te voldoen\autocite{Kleuskens2026}.

Kortom, traditionele handmatige audits in RACF-omgevingen worden beperkt door hun arbeidsintensieve aard, afhankelijkheid van expertise en gebrek aan schaalbaarheid. Deze beperkingen benadrukken de noodzaak van technische en reproduceerbare methoden voor het auditen van gebruikersrechten en de vroege detectie van privilege creep.

\section{Privilege Creep: Concept en Risico's}
\subsection{Definitie}
Privilege creep verwijst naar de geleidelijke accumulatie van toegangsrechten die een gebruiker niet langer functioneel nodig heeft voor hun huidige rol. Het is een fenomeen waarbij gebruikers na verloop van tijd meer machtigingen accumuleren dan strikt noodzakelijk, als gevolg van rolwijzigingen, tijdelijke taken of historische configuraties\autocite{Boorman2026}. In mainframe-omgevingen zoals IBM z/OS is privilege creep moeilijk te detecteren omdat rechten niet direct zichtbaar zijn en zowel directe als geërfde toegangsrechten bestaan\autocite{Kanvar2026}.
Het concept is direct gerelateerd aan het principe van de Least privilege, dat voorschrijft dat gebruikers alleen de minimaal noodzakelijke rechten zouden moeten hebben om hun taken uit te voeren\autocite{YuandPeiyangXueandYunhanQuandLeiJu2026}. Wanneer dit principe niet consequent wordt gehandhaafd, ontstaan cumulatieve overmatige rechten, waardoor de kans op ongeautoriseerde toegang, interne fouten of beveiligingsincidenten toeneemt\autocite{Paul2026}. In RACF-omgevingen kunnen situaties van privilege creep optreden door de combinatie van individuele rechten, groepslidmaatschappen en geërfde autorisaties, wat toezicht en beheer complex maakt\autocite{Kanvar2026}.

\subsection{Oorzaken}
De belangrijkste oorzaken van privilege creep zijn gerelateerd aan operationele en managementprocessen binnen organisaties:
\begin{itemize}
	\item Tijdelijk of ad-hoc verleende rechten
	\begin{itemize}
		\item Rechten worden tijdelijk verleend voor specifieke taken of projecten. Als deze rechten niet systematisch worden ingetrokken, blijven ze actief en stapelen ze zich op\autocite{Boorman2026}.
	\end{itemize}
	\item Gebrek aan periodieke evaluatie of automatische intrekking
	\begin{itemize}
		\item In  RACF-omgevingen is er geen geautomatiseerd proces om periodiek de machtigingen te herzien en te verwijderen die niet langer nodig zijn. Dit leidt tot de accumulatie van overtollige rechten in de loop van de tijd\autocite{Kleuskens2026}.
	\end{itemize}
	\item Rolwijzigingen en gebruikerslevenscyclus
	\begin{itemize}
		\item In het geval van functiewijzigingen of interne overplaatsingen, krijgen gebruikers nieuwe rechten, terwijl oude rechten  behouden blijven\autocite{Dau2026}.
	\end{itemize}
	\item Historisch gegroeide configuraties
	\begin{itemize}
		\item Groepsstructuren en middelenallocaties die in de loop der jaren zijn ontwikkeld, kunnen cumulatieve rechten creëren die niet langer relevant zijn voor de huidige bedrijfsprocessen\autocite{Dau2026}.
	\end{itemize}
	\item Complexiteit van hiërarchische groepen en erfelijke rechten
	\begin{itemize}
		\item Rechten via groepslidmaatschappen en hiërarchische structuren kunnen indirect extra toegangsrechten verlenen, waardoor het moeilijk wordt om te bepalen welke rechten functioneel noodzakelijk zijn\autocite{Kanvar2026}.
\end{itemize}	
\end{itemize}
Samenvattend ontstaat privilege creep door een combinatie van operationele , organisatorische veranderingen en technische complexiteit, waardoor gebruikers geleidelijk meer rechten krijgen dan functioneel vereist.

\subsection{Gevolgen}
De gevolgen van privilege creep zijn zowel operationeel als compliance-gerelateerd en kunnen risico's voor organisaties met zich meebrengen. Allereerst leidt privilege creep tot een hoger risico op interne fouten\autocite{Boorman2026}. Gebruikers die meer rechten hebben dan functioneel noodzakelijk, kunnen onbedoeld kritieke middelen wijzigen of verwijderen, datasets overschrijven of transacties uitvoeren buiten hun expertisegebied. Deze fouten zijn te moeilijk te traceren, in complexe mainframe-omgevingen met geërfde rechten en historische configuraties\autocite{Dau2026}. Zelfs routinematige handelingen kunnen leiden tot operationele verstoringen, zoals gegevensverlies, vertragingen in de verwerking van transacties of inconsistenties in systemen die cruciaal zijn voor de missie\autocite{Bernstein2026}.
Bovendien verhoogt privilegecreep het risico op frauduleuze activiteiten\autocite{Paul2026}. Werknemers met buitensporige rechten kunnen opzettelijk systemen manipuleren, ongeautoriseerde transacties uitvoeren of vertrouwelijke gegevens bekijken die niet relevant zijn voor hun rol. Dit kan variëren van kleine misbruiksituaties, zoals toegang tot niet-essentiële financiële gegevens, tot grotere incidenten waarbij gevoelige klant- of bedrijfsinformatie wordt misbruikt of gelekt. Omdat rechten worden geërfd via groepen en hiërarchieën, is het moeilijk te bepalen welke gebruiker daadwerkelijk toegang had tot specifieke middelen op het moment van het incident, wat forensisch onderzoek bemoeilijkt\autocite{Kanvar2026}.

Ten derde verhoogt privilege creep de complexiteit van audits en naleving van regelgeving\autocite{Wagenseil2026}. Organisaties die moeten voldoen aan interne controles of externe wetten en regelgeving, zoals SOX (Sarbanes-Oxley) of ISO 27001, hebben reproduceerbare en transparante toegangscontroleprocessen nodig. Wanneer gebruikers onnodige of overbodige rechten hebben, wordt het moeilijk om aan te tonen dat de toegang consequent beperkt is tot functioneel noodzakelijke rechten. Dit kan leiden tot auditbevindingen, nalevingsschendingen en in specifieke gevallen tot financiële of reputatieschade\autocite{Paul2026}.

Kortom, de accumulatie van overmatige toegangsrechten door privilegecreep vormt een dubbel risico: enerzijds voor de operationele continuïteit door interne fouten, en anderzijds voor de beveiliging en integriteit van bedrijfsgegevens door potentiële fraude. Tegelijkertijd bemoeilijkt het effectieve toegangsbeheer- en controleprocessen, waardoor organisaties onder druk komen te staan om systematische detectie- en beheermethoden te implementeren.

\section{Detectie van Privilege Creep in Mainframes}
\subsection{Bestaande benaderingen}

Bestaande detectiemethoden in mainframe omgevingen bestrijken een breed scala aan technieken, maar ze hebben beperkingen als het gaat om het systematisch en reproduceerbaar identificeren van privilege creep binnen RACF-configuraties\autocite{Kleuskens2026}. Traditionele controles, zoals handmatige audits en beleidscontroles, zijn arbeidsintensief en afhankelijk van individuele kennis, waardoor periodieke en consistente evaluaties moeilijk uit te voeren zijn in grote of complexe omgevingen\autocite{Wagenseil2026}. Dit maakt ze minder toepasbaar voor continue monitoring en objectieve detectie van buitensporige toegangsrechten op grote schaal.
Daarnaast zijn er commerciële en ingebouwde tools die zich richten op bredere beveiligingsmonitoring en anomaliedetectie. Een voorbeeld hiervan is IBM Threat Detection voor z/OS (TDz), een product dat kunstmatige intelligentie gebruikt om anomalieën in gegevens toegangspatronen te detecteren, zoals onverwachte of  kwaadaardige toegangspogingen tot datasets\autocite{Bassett2026}. Het systeem verzamelt toegangsgegevens via SMF-records en genereert waarschuwingen voor afwijkend gedrag op systeemniveau. Hoewel dergelijke oplossingen waardevol zijn voor het identificeren van potentieel kwaadaardige activiteiten als onderdeel van een defense-in-depth strategie, zijn ze niet specifiek ontworpen om privilege creep te detecteren\autocite{Kleuskens2026}. TDz richt zich op anomalieën en verdachte toegangs patronen, niet op het systematisch analyseren van RACF-configuratiegegevens zoals gebruikers-, groeps- en resourceprofielen om overmatige of ongebruikte rechten te identificeren.

Andere tools, zoals de IBM Security zSecure suite, bieden audit- en compliance-functionaliteit en kunnen helpen bij het analyseren van RACF-configuraties en loggegevens, maar ze blijven grotendeels gericht op het ondersteunen van administratieve en compliance-taken, niet op het automatisch identificeren van privilege creep als een specifiek resultaat van data-analyse ‒ bijvoorbeeld door historische en huidige autorisaties te correleren of rechten te identificeren die functioneel niet langer worden gebruikt\autocite{Kleuskens2026}.

Samenvattend zijn bestaande methoden voor privilege-detectie in RACF-omgevingen grotendeels gericht op handmatige beoordelingen en generieke anomaliedetectie of auditondersteuning. Ze richten zich niet op de specifieke kenmerken van privilege creep, zoals het objectief analyseren van geïmplementeerde rechten tegen functionele noodzaak, en zijn daarom beperkt in hun toepasbaarheid voor grootschalige, reproduceerbare detectie in complexe RACF-omgevingen.

\subsection{Relevantie voor RACF}
RACF gegevens zijn complex vanwege hiërarchische groepen en uitgebreide autorisatiestructuren\autocite{Kanvar2026}. Dit maakt handmatige detectie van privilege creep inefficiënt en foutgevoelig, wat de noodzaak benadrukt voor een technische, reproduceerbare methode. Gebruikers kunnen rechten direct of via meerdere lagen van groepslidmaatschappen toegewezen krijgen, waarbij ze rechten erven van bovenliggende groepen. In grote organisaties betekent dit dat een gebruiker potentieel honderden expliciete en impliciete toegangsrechten kan hebben, waarvan niet noodzakelijk zijn\autocite{Dau2026}.
Bovendien veranderen zowel gebruikersrollen als resourceconfiguraties continu. Zonder geautomatiseerde analyse is het vrijwel onmogelijk om effectief te bepalen welke rechten actief worden gebruikt en welke verouderd zijn\autocite{Boorman2026}. Traditionele audits en handmatige controles missen de schaal en precisie om dit op een consistente en reproduceerbare manier te doen\autocite{Wagenseil2026}. Dit verhoogt niet alleen het risico van overmatige machtigingen, maar maakt het ook moeilijk om naleving en het principe van de Least privileges aan te tonen tijdens interne of externe audits\autocite{YuandPeiyangXueandYunhanQuandLeiJu2026}.

Technische detectiemethoden bieden hier een oplossing door RACF-gegevens systematisch te analyseren, gebruikersprofielen, groepsstructuren en resource-autorisaties te correleren en te visualiseren\autocite{Kanvar2026}. Door algoritmen toe te passen die redundante of ongebruikte rechten identificeren, kan een objectief beeld worden gevormd van het effectieve toegangsprofiel van gebruikers. Dit vermindert menselijke fouten, versnelt het detectieproces en ondersteunt reproduceerbare evaluaties van toegangsrechten.

Kortom, de inherente complexiteit van RACF-gegevens en de dynamiek van toegangsrechten maken handmatige detectie ontoereikend, en benadrukken het belang van een systematische, technisch ondersteunde aanpak om privilege creep betrouwbaar te identificeren en te beheren.

\section{Onderzoekskloof}

Ondanks bestaande methoden is er een duidelijke kloof: er ontbreekt een objectieve, systematische en technisch reproduceerbare methode voor het detecteren van privilege creep binnen RACF-omgevingen. Traditionele benaderingen, zoals handmatige audits, beleidscontroles en rolanalyses, zijn afhankelijk van individuele expertise en kennis van de complexe hiërarchische groepsstructuren en resourceconfiguraties. Dit maakt audits tijdrovend, inconsistent en moeilijk schaalbaar voor grote organisaties met honderden of duizenden gebruikers.
Bovendien bieden moderne mainframe-beveiligingstools, zoals IBM Threat Detection for z/OS (TDz), mogelijkheden voor anomaliedetectie en monitoring van verdachte activiteiten, maar ze zijn niet specifiek ontworpen voor het systematisch analyseren van RACF-gegevens om overmatige of ongebruikte machtigingen te identificeren. Deze oplossingen richten zich voornamelijk op realtime dreigingsdetectie en niet op de objectieve evaluatie van de functionele noodzaak van toegewezen rechten. Als gevolg hiervan blijft privilegecreep  onopgemerkt, ondanks de aanwezigheid van uitgebreide monitoringinfrastructuur.

Commerciële audit- en compliance-tools, zoals de IBM zSecure-suite, helpen voornamelijk bij het genereren van rapporten en het uitvoeren van periodieke controles, maar ze bieden geen reproduceerbare methode voor het automatisch identificeren van rechten die niet langer functioneel nodig zijn. Het ontbreken van een dergelijke methode leidt tot een kritieke kloof in toegangsbeheer: organisaties kunnen inderdaad controleren of rechten volgens het beleid zijn toegewezen, maar ze hebben geen systematische manier om te bepalen of deze rechten daadwerkelijk in gebruik zijn of overbodig zijn geworden.

Deze kloof vormt de kern van het onderzoeksprobleem van deze thesis: er is behoefte aan een technische, datagestuurde methode die RACF-gebruikersprofielen, groepsstructuren en resource-autorisaties kan analyseren om privilege creep objectief en reproduceerbaar te detecteren. Een dergelijke methode zou niet alleen de efficiëntie en nauwkeurigheid van toegangscontroles verbeteren, maar ook bijdragen aan naleving, auditvoorbereiding en de handhaving van het principe van de Least privilege binnen mainframe-omgevingen.

Kortom, de bestaande methoden en tools bieden waardevolle ondersteuning voor toegangsbeheer en beveiliging, maar er ontbreekt een geautomatiseerde en systematische aanpak die specifiek is afgestemd op het detecteren van privilege creep in RACF. Het ontwikkelen van een dergelijke aanpak is daarom het centrale doel van deze bachelorproef.

\section{Samenvatting}
De analyse van de huidige situatie toont aan dat de complexiteit van RACF-configuraties en historisch gegroeide rechtenstructuren een belangrijke factor zijn bij het ontstaan van privilege creep. In grote organisaties met langdurig actieve gebruikers en hiërarchische groepsstructuren kunnen toegangsrechten geleidelijk accumuleren, waardoor gebruikers meer machtigingen hebben dan functioneel noodzakelijk. Dit verhoogt niet alleen het risico op interne fouten en fraude, maar bemoeilijkt ook audits en naleving van regelgeving, aangezien het moeilijk is om het effectieve toegangsprofiel van gebruikers volledig te overzien.
Bestaande methoden voor toegangscontrole en detectie van overmatige rechten, zoals handmatige audits, rolmining en compliance-tools, blijken onvoldoende systematisch en reproduceerbaar te zijn. Handmatige processen zijn foutgevoelig, tijdrovend en afhankelijk van individuele expertise, terwijl tools zoals IBM Threat Detection voor z/OS zich richten op anomaliedetectie en niet specifiek op privilege creep. Als gevolg hiervan blijven onnodige machtigingen onopgemerkt, en is er geen betrouwbare methode om te bepalen welke machtigingen daadwerkelijk functioneel noodzakelijk zijn.

Deze bevindingen benadrukken de noodzaak van een technische, datagestuurde aanpak die RACF-gegevens systematisch analyseert. Door gebruikersprofielen, groepsstructuren en resource-autorisaties te correleren, kan een objectief inzicht worden verkregen in welke toegangsrechten functioneel relevant zijn en welke  overbodig zijn. Een dergelijke methode biedt niet alleen een oplossing voor de huidige beperkingen van handmatige en op beleid gebaseerde controles, maar ondersteunt ook de toepassing van het principe van de Least privilege, verbetert de efficiëntie van audits en draagt bij aan het minimaliseren van operationele en beveiligingsrisico's.

Samenvattend creëert de combinatie van historisch gegroeide rechten, complexe hiërarchieën en de beperkingen van bestaande detectiemethoden een duidelijke onderzoekskloof. Dit onderstreept het belang van deze bachelorproef, waarin een Proof of Concept wordt ontwikkeld om privilege creep systematisch en reproduceerbaar te detecteren, zodat het toegangsbeheer binnen RACF-omgevingen kan worden geoptimaliseerd.



