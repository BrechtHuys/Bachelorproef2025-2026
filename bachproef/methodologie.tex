%%=============================================================================
%% Methodologie
%%=============================================================================

\chapter{\IfLanguageName{dutch}{Methodologie}{Methodology}}%
\label{ch:methodologie}
Deze bachelorproef hanteert een toegepaste onderzoeksbenadering, waarbij zowel analytische als technische onderzoeksmethoden worden gecombineerd om het probleem van privilege creep in RACF-omgevingen te onderzoeken en een concrete oplossing te ontwikkelen. Het onderzoek is verdeeld in drie duidelijk gedefinieerde fasen: een analysefase, een ontwikkelings- en testfase, en een synthese- en evaluatiefase. Deze fasen zijn ontworpen om zowel het probleemgebied als het oplossingsgebied systematisch aan te pakken, en om een reproduceerbare, technisch onderbouwde aanpak te bereiken.

\section{Analysefase Probleemgebied}
In de analysefase wordt inzicht verkregen in het ontstaan, de impact en de mechanismen van privilege creep binnen RACF-omgevingen. Hiervoor worden drie onderzoeksmethoden gecombineerd: literatuuronderzoek, documentanalyse en interviews met belanghebbenden.
\begin{itemize}
	\item Literatuuronderzoek: Academische en technische bronnen worden bestudeerd met betrekking tot privilege creep, het principe van de Least privilege, en Identity & Access Management binnen mainframe-omgevingen. Dit biedt een theoretisch kader voor het begrijpen van de oorzaken en gevolgen van overmatige privileges.
	\item Documentanalyse: RACF- en z/OS-gerelateerde documentatie wordt onderzocht, inclusief:
	\begin{itemize}
		\item IBM RACF Beheerder van Beveiliging Gids
		\item Compliance- en auditrichtlijnen met betrekking tot toegangsbeheer en het principe van de Least privileges	
		\item RACF Administratiehandleiding
	\end{itemize}
	\item Technische analyse van RACF-gegevens: RACF-extracts en configuratiegegevens worden onderzocht om inzicht te krijgen in gebruikersstructuren, groepshiërarchieën en toegewezen autorisaties. De focus ligt op het identificeren van patronen zoals:
	\begin{itemize}
		\item Gebruikers met overlappende groepslidmaatschappen	
		\item Langdurig verleende privileges zonder duidelijke functionele noodzaak	
		\item Afwijkingen van het principe van de Least privileges
	\end{itemize}
	Stakeholderinterviews: Semi-gestructureerde interviews worden afgenomen met RACF-beveiligingsbeheerders en auditmanagers om de technische bevindingen te contextualiseren. De interviews richten zich op:
	\begin{itemize}
		\item Huidige processen voor het verlenen en intrekken van rechten	
		\item Knopen in toegangsevaluaties en audits	
		\item Ervaringen en verwachtingen met betrekking tot geautomatiseerde detectie van privilege creep
	\end{itemize}
\end{itemize}

\section{Ontwikkelings- en Testfase Oplossingsdomein}
In deze fase wordt er een technische Proof of Concept (PoC) ontwikkeld om privilege creep systematisch te detecteren. De PoC analyseert specifieke RACF-gegevens , zoals:
\begin{itemize}
	\item Gebruikersprofielen (GEBRUIKERSID, SPECIAAL, OPERATIES, AUDITOR)
	\item Groepslidmaatschappen en hiërarchieën
	\item Dataset- en algemene bronnenprofielen met bijbehorende autorisaties
	\item Beschikbare activiteit of laatste gebruiksinformatie
\end{itemize}
De PoC is geïmplementeerd als een zelfstandige script of tool die RACF-exportgegevens verwerkt. De nadruk ligt op reproduceerbaarheid en transparantie, zodat de analyse herhaalbaar is en de resultaten begrijpelijk blijven voor RACF-beheerders. De output bestaat uit gestructureerde rapporten die potentiële gevallen van privilege creep per gebruiker weergeven.
\begin{itemize}
	\item De PoC wordt getest in een gecontroleerde z/OS-testomgeving. De testfase omvat representatieve scenario's, waaronder:
	\item Gebruikers met privileges van  groepen die niet langer overeenkomen met hun rol
	\item Gebruikers met verhoogde privileges die lange tijd niet zijn gebruikt
\end{itemize}

\section{Synthese en Evaluatie Fase}
Synthese- en Evaluatiefase In de laatste fase worden de resultaten van de analyse- en ontwikkelingsfases samengebracht. De werking van de PoC wordt geëvalueerd op:
\begin{itemize}
	\item Correctheid: detecteert de tool nauwkeurig gevallen van privilege creep?
	\item Reproduceerbaarheid: levert de analyse identieke resultaten op bij herhaling?
	\item Praktische toepasbaarheid: is de output interpreteerbaar en bruikbaar voor RACF-beheerders en auditors?
\end{itemize}
Op basis van deze evaluatie worden concrete aanbevelingen geformuleerd om privilege creep binnen RACF te beheersen, met de nadruk op periodieke toegangsbeoordelingen en technische ondersteuning voor auditprocessen.
\section{Vereisteanalyse}
De functionele en niet-functionele vereisten van de PoC zijn afgeleid van literatuur, documentatie en verwachtingen van belanghebbenden.
Doel: Het ondersteunen van RACF-beheerders en auditors bij het identificeren van historisch gegroeide, overbodige of risicovolle autorisaties.

Functionele vereisten (FR):
\begin{itemize}
	\item FR1: Het analyseren van gebruikersprofielen inclusief attributen
	\item FR2: Het in kaart brengen van groepslidmaatschappen en hiërarchieën
	\item FR3: Het genereren van gestructureerde rapporten per gebruiker
\end{itemize}
Niet-functionele vereisten (NFR):
\begin{itemize}
	\item NFR1: Reproduceerbare analyses
	\item NFR2: Transparantie en interpreteerbaarheid voor beheerders
	\item NFR3: Geen wijzigingen aan RACF-configuraties of productiedata
	\item NFR4: Schaalbaarheid naar grote RACF-omgevingen
	\item NFR5: Naleving van beveiligings- en compliance-richtlijnen
\end{itemize}
Voorwaarden en aannames: toegang tot representatieve RACF-exportgegevens, focus op detectie en rapportage, geen GUI, proof-of-concept gericht op technische haalbaarheid.

Scope: geen realtime monitoring, geen automatische intrekking van autorisaties, of integratie met externe IAM-platforms.

\section{Fasering van Planning}
\begin{itemize}
	\item Fase 1  Analyse (februari 2026): literatuuronderzoek, documentanalyse, identificatie van relevante data-elementen
	\item Fase 2  Ontwerp (maart 2026): functioneel en technisch ontwerp van PoC, datamodellen en rapportagestructuur
	\item Fase 3  Implementatie (april 2026): ontwikkeling van PoC en verwerking van RACF-gegevens
	\item Fase 4  Testen en evaluatie (mei 2026): uitvoering van scenario's, evaluatie van juistheid en reproduceerbaarheid
	\item Fase 5  Synthese en Rapportage (mei 2026): resultaten analyseren, aanbevelingen formuleren, bachelorthesis afronden
\end{itemize}
%% TODO: In dit hoofstuk geef je een korte toelichting over hoe je te werk bent
%% gegaan. Verdeel je onderzoek in grote fasen, en licht in elke fase toe wat
%% de doelstelling was, welke deliverables daar uit gekomen zijn, en welke
%% onderzoeksmethoden je daarbij toegepast hebt. Verantwoord waarom je
%% op deze manier te werk gegaan bent.
%% 
%% Voorbeelden van zulke fasen zijn: literatuurstudie, opstellen van een
%% requirements-analyse, opstellen long-list (bij vergelijkende studie),
%% selectie van geschikte tools (bij vergelijkende studie, "short-list"),
%% opzetten testopstelling/PoC, uitvoeren testen en verzamelen
%% van resultaten, analyse van resultaten, ...
%%
%% !!!!! LET OP !!!!!
%%
%% Het is uitdrukkelijk NIET de bedoeling dat je het grootste deel van de corpus
%% van je bachelorproef in dit hoofstuk verwerkt! Dit hoofdstuk is eerder een
%% kort overzicht van je plan van aanpak.
%%
%% Maak voor elke fase (behalve het literatuuronderzoek) een NIEUW HOOFDSTUK aan
%% en geef het een gepaste titel.


